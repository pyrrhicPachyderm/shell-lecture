\documentclass[pdf,usenames,dvipsnames,14pt]{beamer}%usenames and dvipsnames are for colours.
\mode<presentation>{}
\usetheme{Madrid}
\usecolortheme{seahorse}%seagull?

%\setbeamertemplate{title page}[default][colsep=-4bp,rounded=true] %Remove drop shadow on title page box
%\setbeamertemplate{frametitle}[default][center] %Centre the titles of frames

\usepackage[T1]{fontenc}
\usepackage{lmodern} %Makes \textasciitilde not raised.
\usepackage{inconsolata} %A nice monospace font
\newcommand\hyphen{\char`\-}
\newcommand\textasciicaret{\textasciicircum}

\usepackage{siunitx}

\usepackage{xcolor}
\usepackage{tikz}
\usetikzlibrary{positioning,arrows.meta,calc,backgrounds,intersections,shapes}

\usepackage{caption}
\usepackage{hyperref}
\hypersetup{
	colorlinks=true,
	urlcolor=blue,
}
\urlstyle{same} %URLs are set in the font of the surrounding text

%\definecolor{GoodGreen}{RGB}{0,140,0}

\graphicspath{{./figures/}}

\renewcommand\emph[1]{\textbf{#1}}

\title[Command Line \& Shell]{The Command Line, Shell and Shell Scripting}
%\subtitle{\vspace{2mm}\small{
%	based upon the paper\\
%	\emph{Genomic linkage of male song and female acoustic preference QTL underlying a rapid species radiation}\\
%	by K.\ Shaw and S.\ Lesnick (2009)\\
%	in \textit{PNAS}
%}}
\author{Christopher Brown}
\date{}

\begin{document}

% title frame
\begin{frame}
	\titlepage
\end{frame}

\section{Introduction}

\begin{frame}{What is the Shell?}
	\begin{columns}[t]
	\column{0.5\textwidth}
		\colorlet{shellcol2}{black!30}
		\colorlet{shellcol1}{white}
		\begin{tikzpicture}
			%Circles
			\draw [fill=shellcol2] (0,0) circle (3.5);
			\draw [fill=shellcol1] (0,0) circle (2.5);
			\draw [fill=shellcol2] (0,0) circle (1.5);
			%Labels
			\node at (0,2.95) {Shell};
			\node at (0,1.9) {Kernel};
			\node at (0,0) {Hardware};
		\end{tikzpicture}
	\column{0.3\textwidth}
		Interface between user and operating system kernel.
	\end{columns}
	\note{
		Being replaced by Graphical User Interfaces more these days.
	}
\end{frame}

\begin{frame}{Why Shell?}
	\begin{itemize}
		\item Automisation
		\item Reproducibility
		\item Scripting
		\begin{itemize}
			\item Call other programs
			\item Manipulate files
			\item String utilities
		\end{itemize}
	\end{itemize}
\end{frame}

\begin{frame}{Which Shell?}
	\begin{itemize}
		\item \texttt{sh}: the Bourne Shell by Stephen Bourne, 1977
		\item \texttt{bash}: the Bourne Again SHell, 1989
		\item \texttt{zsh}, \texttt{ksh}, \texttt{tcsh}, \texttt{rc}\dots
	\end{itemize}
	We will use \texttt{bash}.
	
	Default on most Linux distros and Mac OS X.
	
	On Windows via Cygwin or WSL.
\end{frame}

\section{Basic Commands}

\begin{frame}{UNIX Directories}
	\begin{itemize}
		\item \texttt{/} is the root directory
		\item \texttt{\textasciitilde} is your user's home directory: \texttt{/home/abc123/}
		\item \texttt{.} is the current directory
		\item \texttt{..} moves up one directory
	\end{itemize}
	Paths starting with \texttt{/} or \texttt{\textasciitilde} are \emph{absolute paths}.
	
	Others are \emph{relative paths}, and start from the current directory.
\end{frame}

\begin{frame}{Moving Around}
	\begin{itemize}
		\item \texttt{pwd}: Print Working Directory
		\item \texttt{cd}: Change Directory
		\begin{itemize}
			\item \texttt{cd \hyphen} goes to previous directory
		\end{itemize}
	\end{itemize}
\end{frame}

\begin{frame}{What's Here?}
	\begin{itemize}
		\item \texttt{ls}: LiSt
		\begin{itemize}
			\item \texttt{ls \hyphen l} gives extra information
			\item \texttt{ls \hyphen a} includes hidden files/folders
		\end{itemize}
	\end{itemize}
\end{frame}

\begin{frame}{Changing Stuff}
	\begin{itemize}
		\item \texttt{mkdir}: MaKe DIRectory
		\item \texttt{rmdir}: ReMove empty DIRectory
	\end{itemize}
\end{frame}

\begin{frame}{More Changing Stuff}
	\begin{itemize}
		\item \texttt{touch}: create a blank file, change last modified date
		\item \texttt{cp}: CoPy a file/directory
		\item \texttt{mv}: MoVe/rename a file/directory
		\item \texttt{rm}: ReMove a file
		\begin{itemize}
			\item \texttt{rm \hyphen i} asks confirmation: safer
			\item \texttt{rm \hyphen r} removes a directory and everything inside
		\end{itemize}
	\end{itemize}
	\note{
		There is no trash for \texttt{rm}!
	}
\end{frame}

\begin{frame}{Viewing Files}
	\begin{itemize}
		\item \texttt{cat}: print a file, or conCATenate several
		\item \texttt{more}: view a file interactively
		\item \texttt{less}: \texttt{more}, but better
	\end{itemize}
	\note{
		There is no trash for \texttt{rm}!
	}
\end{frame}

\begin{frame}{Text Editors}
	\begin{itemize}
		\item \texttt{nano}: simple, useful for quick stuff
		\begin{itemize}
			\item \texttt{\textasciicaret S} to save
			\item \texttt{\textasciicaret X} to exit
		\end{itemize}
		\item \texttt{vim}, \texttt{emacs}: super complicated \& super powerful
	\end{itemize}
\end{frame}

\section{How Stuff Works}

\end{document}
